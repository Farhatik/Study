\documentclass{article} % добавить leqno в [] для нумерации слева

%%% Работа с русским языком
\usepackage{cmap}					% поиск в PDF
\usepackage{mathtext} 				% русские буквы в фомулах
\usepackage[T2A]{fontenc}			% кодировка
\usepackage[utf8]{inputenc}			% кодировка исходного текста
\usepackage[english,russian]{babel}	% локализация и переносы

%%% Дополнительная работа с математикой
\usepackage{amsmath,amsfonts,amssymb,amsthm,mathtools} % AMS
\usepackage{icomma} % "Умная" запятая: $0,2$ --- число, $0, 2$ --- перечисление

%% Номера формул
%%\mathtoolsset{showonlyrefs=false} % Показывать номера только у тех формул, на которые есть \eqref{} в тексте.

%% Шрифты
\usepackage{euscript}	 % Шрифт Евклид
\usepackage{mathrsfs} % Красивый матшрифт
\usepackage{graphicx}

\newtheorem{theorem}{Теорема}
\newtheorem{definition}{Определение}
\newtheorem{lemma}[theorem]{Лемма}

%% Перенос знаков в формулах (по Львовскому)
\newcommand*{\hm}[1]{#1\nobreak\discretionary{}
{\hbox{$\mathsurround=0pt #1$}}{}}


%%% Заголовок
\author{Агаев Фархат}
\title{Доказательство Теоремы Перрона-Фробениуса}
\date{\today}

\begin{document} % конец преамбулы, начало документа


\maketitle

\section*{Определения}

\section{Теорема Перрона-Фробениуса для неприводимых матриц}

\begin{theorem}
    Пусть матрица $A \geq 0$ и неприводима со спектральным радиусом r. 
    Тогда следующие утверждения верны:
    \begin{enumerate}
        \it r - это собственное значения для А;
        \item Алгебраическая и геометрическая кратность r равна 1;
        \item r обладает положительным собственным вектором;
        \item Все собственные значения A с абсолютным значением r имеют кратность 1; 
        если их кратность h, тогда они являются
         (комплексными) решениями уравнения $\lambda^h = r^h$
        \item cпектр A, рассматриваемый как мультимножество,
         отображается сам по себе 
        при вращении комплекса плоскость под углом 2
        \item  если h > 1 , то для некоторой перестановки 
        $\pi {\it one has} \pi(A) = \left\{\begin{array}{lllll}
            0 & B_{1} & 0 & \cdots & 0\\
            0 & 0 & B_{2} & \cdots & 0\\
             &  &  & \ddots & \\
            0\cdots & 0 & 0 & \cdots & B_{h-1}\\
            B_{h} & 0 & 0 & \cdots & 0
            \end{array}\right\}$
         где все блоки по главной диагонали имеют квадрат 2
    \end{enumerate}
\end{theorem}
\begin{lemma}
    Пусть A ≥ 0 - неприводимая матрица. 
\end{lemma}
Тогда (A + I) n − 1 > 0 . Доказательство. Отметим, что матрица B положительна, 
если вектор B x для любого x ≥ 0 , x = 0 положителен 3 . 
Сравните число ненулевых координат произвольного ненулевого вектора 
x ≥ 0 и вектор y = (A + I) x . Если x j > 0 , то y j = (A x) j + x j > 0 . 
Таким образом, множество нулевых координат у является 
подмножеством такого набора для х . 
Предположим, что эти два набора совпадают. 
Измените основу с помощью перестановка π такая, 
что новые координаты x и y равны [ u 0 ] и [ v 0 ] соответственно,
где u, v> 0 имеют одинаковый размер. Представлять π (A) 
как блочную матрицу [ BC DF ] , где B, F квадратные матрицы,
а размер B равен размеру u . Теперь равенство [ До н.э DF ] 
[ ты 0 ] = [ v 0 ] подразумевает D = 0, потому что u> 0 . 
Тогда A сводится по замечанию 1, что противоречит условиям леммы.
Следовательно, у имеет строго меньше нулевых координат, 
чем х . Применяя этот факт к векторам x , (A + I) x , ..., (A + I) n − 2 x
 и учитывая, что число из нулевых координат x не больше n - 1 , 
 получаем (A + I) n − 1 x> 0 , откуда и результат

\end{document}