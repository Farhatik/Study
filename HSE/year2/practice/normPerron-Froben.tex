\documentclass{article} % добавить leqno в [] для нумерации слева

%%% Работа с русским языком
\usepackage{cmap}					% поиск в PDF
\usepackage{mathtext} 				% русские буквы в фомулах
\usepackage[T2A]{fontenc}			% кодировка
\usepackage[utf8]{inputenc}			% кодировка исходного текста
\usepackage[english,russian]{babel}	% локализация и переносы

%%% Дополнительная работа с математикой
\usepackage{amsmath,amsfonts,amssymb,amsthm,mathtools} % AMS
\usepackage{icomma} % "Умная" запятая: $0,2$ --- число, $0, 2$ --- перечисление

%% Номера формул
%%\mathtoolsset{showonlyrefs=false} % Показывать номера только у тех формул, на которые есть \eqref{} в тексте.

%% Шрифты
\usepackage{euscript}	 % Шрифт Евклид
\usepackage{mathrsfs} % Красивый матшрифт
\usepackage{graphicx}

\newtheorem{theorem}{Теорема}
\newtheorem{definition}{Определение}
\newtheorem{lemma}[theorem]{Лемма}

%% Перенос знаков в формулах (по Львовскому)
\newcommand*{\hm}[1]{#1\nobreak\discretionary{}
{\hbox{$\mathsurround=0pt #1$}}{}}


%%% Заголовок
\author{Агаев Фархат}
\title{Доказательство Теоремы Перрона-Фробениуса}
\date{\today}

\begin{document} % конец преамбулы, начало документа


\maketitle

\section*{Определения}

\subsubsection*{Теорема Перрона-Фробениуса для неприводимых матриц}
    Пусть матрица $A \geq 0$ и неприводима со спектральным радиусом r. 
    Тогда следующие утверждения верны:
    \begin{enumerate}
        \item r - это собственное значения для А;
        \item Алгебраическая и геометрическая кратность r равна 1;
        \item r обладает положительным собственным вектором;
        \item Все собственные значения A с абсолютным значением r имеют кратность 1; 
        если их кратность h, тогда они являются
        (комплексными) решениями уравнения $\lambda^h = r^h$
        \item спектр матрицы А, рассматриваемый как мультимножество, 
        отображается сам по себе при вращении комплекса
    
    \end{enumerate}



\begin{lemma}
    Пусть $A \geq 0$- неприводимая матрица. 
\end{lemma}



\end{document}