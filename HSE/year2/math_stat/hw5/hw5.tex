\documentclass[a4paper, 12pt]{article} % добавить leqno в [] для нумерации слева


%%% Работа с русским языком
\usepackage{cmap}					% поиск в PDF
\usepackage{mathtext} 				% русские буквы в фомулах
\usepackage[T2A]{fontenc}			% кодировка
\usepackage[utf8]{inputenc}			% кодировка исходного текста
\usepackage[english,russian]{babel}	% локализация и переносы

%%% Дополнительная работа с математикой
\usepackage{amsmath,amsfonts,amssymb,amsthm,mathtools} % AMS
\usepackage{icomma} % "Умная" запятая: $0,2$ --- число, $0, 2$ --- перечисление

%% Номера формулx
\mathtoolsset{showonlyrefs=true} % Показывать номера только у тех формул, на которые есть \eqref{} в тексте.

%% Шрифты
\usepackage{euscript}	 % Шрифт Евклидxwxx
\usepackage{mathrsfs} % Красивый матшрифт
\usepackage{ dsfont }
%% Свои команды
\DeclareMathOperator{\sgn}{\mathop{sgn}}
\newcommand{\Expect}{\mathsf{E}}

%% Перенос знаков в формулах (по Львовскому)
\newcommand*{\hm}[1]{#1\nobreak\discretionary{}
{\hbox{$\mathsurround=0pt #1$}}{}}

%%% Заголовок
\author{Агаев Фархат}
\title{Домашнее задание по мат стату №6}
\date{\today}

\begin{document} % конец преамбулы, начало документа

\maketitle

\section*{Задача 11}
\[ R = \begin{pmatrix}
    4 & 0\\
    0&  4
  \end{pmatrix} \;
   \;
   16 = \begin{vmatrix}
    4 & 0\\
    0&  4
  \end{vmatrix} \;
   \;
  R^{-1} = \begin{pmatrix}
    1/4 & 0\\
    0&  1/4
  \end{pmatrix}
  \;
   \;
  1/16 = \begin{vmatrix}
    2/4 & 0\\
    0&  1/3
  \end{vmatrix}
  \]

  Воспользуемся стандартным приемом о посчитаем с помощью замены на полярные координаты

  \[P(4 \leq \xi^2 + \eta^2 \leq 9) = \frac{1}{8\pi}\iint_{4 \leq x_1^2 + x_2^2 \leq 9} e ^{\frac{x_1^2 + x_2^2}{-8}}dx_1dx_2 = \int_{0}^{2\pi} d\phi  \int_{2}^{3} \frac{-1}{8\pi} e ^{\frac{r^2}{8}}rdr = e^{\frac{-1}{2}} - e^{\frac{-9}{8}}
  \]

Ответ 
\[e^{\frac{-1}{2}} - e^{\frac{-9}{8}}\]

\end{document}
