\documentclass[a4paper, 12pt]{article} % добавить leqno в [] для нумерации слева


%%% Работа с русским языком
\usepackage{cmap}					% поиск в PDF
\usepackage{mathtext} 				% русские буквы в фомулах
\usepackage[T2A]{fontenc}			% кодировка
\usepackage[utf8]{inputenc}			% кодировка исходного текста
\usepackage[english,russian]{babel}	% локализация и переносы

%%% Дополнительная работа с математикой
\usepackage{amsmath,amsfonts,amssymb,amsthm,mathtools} % AMS
\usepackage{icomma} % "Умная" запятая: $0,2$ --- число, $0, 2$ --- перечисление

%% Номера формул
\mathtoolsset{showonlyrefs=true} % Показывать номера только у тех формул, на которые есть \eqref{} в тексте.

%% Шрифты
\usepackage{euscript}	 % Шрифт Евклид
\usepackage{mathrsfs} % Красивый матшрифт

%% Свои команды
\DeclareMathOperator{\sgn}{\mathop{sgn}}
\newcommand{\Expect}{\mathsf{E}}

%% Перенос знаков в формулах (по Львовскому)
\newcommand*{\hm}[1]{#1\nobreak\discretionary{}
{\hbox{$\mathsurround=0pt #1$}}{}}

%%% Заголовок
\author{Агаев Фархат}
\title{Домашнее задание по мат стату №1}
\date{\today}

\begin{document} % конец преамбулы, начало документа

\maketitle
\section*{Задача №9}
Пусть $\xi = \xi_1 + \xi_2 + \xi_3 + \xi_4$ - время приема четырех людей (оно занимает от 4 до 16 минут)
Неравенство Чебышева воспользуемся им 
\[
    P(\xi > 12) < 
 \frac{\Expect \xi}{12} 
\]
Посчитаем по формуле $\Expect \xi = \frac{a + b}{2} = 10$
Получаем такую оценку 
\[
    P(\xi > 12) < 
 \frac{5}{6} 
\]
\section*{Задача №10} {
\[
m_n = min\{\xi_1, \ldots, \xi_n\}    
\]    
\[
    P(m_n > t) = (1-t)^n
\] 
\[
    F = 0, t \leq 0 
\]
\[
    F = 1, t \geq 1
\]
\[
  F_{m_n}(t) = P(m_n \leq t) = 1 - (1-t)^n 
\]

РАсммотрим последовательность случайных величин $m_1, m_2, \ldots$
Очевидно, что последовательность монотонно убывает ($m_1 \geq m_2$ \ldots ) и ограничена. Так как
$ A_1 = min\{\xi_1\} \in A_2 = \min\{\xi_1, \xi_2\}$ \ldots  Очев огр, так как $0 \leq m_n \leq 1$
$\Rightarrow \exists \lim\limits_{n\to \infty}\xi_n $
воспользуемся стандартным приемом сверху мы обозначили события $A_i$
\[
    P(A = \cap_{n = 1}A_n) \rightarrow 0 = 1 - (1 - t)^n = 1 \text{ при } n \rightarrow \infty
\]
Ответ: Почти наверное стремится к нулю (доказывали лемму на семе)
\section*{Задача №11}
    возьмем случ вел. 
\[
    \mu = e^{ln(\eta_n)}
\]
\[
    \mu = e^{ln(\sqrt[n]{\xi_1 \cdots \xi_n})} = e^{\frac{ln(\xi_1) + ln(\xi_2) + \ldots + ln(\xi_n)}{n}}
    \]
    По закону больших чисел в слабой форме ($\Expect|ln (\xi_k)|^4$ конечное)
    \[
    \Expect \xi = -1    
    \]
    Так как 
    \[
      \int_{0}^{1} ln(x) dx  = xln(x) - x |^1_0
    \]
    $\Rightarrow $
    Ответ: \[
        \frac{1}{e}
        \]
\end{document}
