\documentclass[a4paper, 12pt]{article} % добавить leqno в [] для нумерации слева


%%% Работа с русским языком
\usepackage{cmap}					% поиск в PDF
\usepackage{mathtext} 				% русские буквы в фомулах
\usepackage[T2A]{fontenc}			% кодировка
\usepackage[utf8]{inputenc}			% кодировка исходного текста
\usepackage[english,russian]{babel}	% локализация и переносы

%%% Дополнительная работа с математикой
\usepackage{amsmath,amsfonts,amssymb,amsthm,mathtools} % AMS
\usepackage{icomma} % "Умная" запятая: $0,2$ --- число, $0, 2$ --- перечисление

%% Номера формул
\mathtoolsset{showonlyrefs=true} % Показывать номера только у тех формул, на которые есть \eqref{} в тексте.

%% Шрифты
\usepackage{euscript}	 % Шрифт Евклид
\usepackage{mathrsfs} % Красивый матшрифт
\usepackage{ dsfont }
%% Свои команды
\DeclareMathOperator{\sgn}{\mathop{sgn}}
\newcommand{\Expect}{\mathsf{E}}

%% Перенос знаков в формулах (по Львовскому)
\newcommand*{\hm}[1]{#1\nobreak\discretionary{}
{\hbox{$\mathsurround=0pt #1$}}{}}

%%% Заголовок
\author{Агаев Фархат}
\title{Домашнее задание по мат стату №3}
\date{\today}

\begin{document} % конец преамбулы, начало документа

\maketitle

\section*{Задача 4c}
Решали на семе
\[\gamma = 
\frac{\xi + \zeta \cdot \eta}{\sqrt{1 + \zeta^2}}\]

\[\varphi_{\gamma} = \mathds{E} e^{it\gamma} 
= \int\int\int e^{it\frac{x + z\cdot y}{\sqrt{1 + z^2}}} \cdot \rho(x) \rho(y) \rho(z) dx dy dz 
\]
 \[= \int \mathds{E} e^{it\frac{\xi + \eta \cdot z}{\sqrt{1 + z^2}}} \rho(z) dz =
  \int \mathds{E} e^{\frac{it\xi}{\sqrt{1 + z^2}}} \cdot e ^ {\frac{izt\eta}{\sqrt{1 + z^2}}}
   \cdot \rho(z) dz
  = \] 

  \[= \int e^{\frac{-1}{2}\frac{t^2}{1 + z^2}}
  \cdot e^{\frac{-1}{2}\frac{t^2z^2t}{1 + z^2}} 
  \rho(z) dz = e^{\frac{-t^2}{2}} \cdot 1 
   = e^{\frac{-t^2}{2}}
   \]


Ответ: $e^{\frac{-t^2}{2}}$

\section*{Задача 9}



Мы знаем, что  $\xi_i$ независымые и так же
что при умножение на константу мы увеличиваем мат ожидание в константу раз
а дисперсия увеличивается в квадрат константы раз
то есть например 
\[2\xi_1, \; N(0, 4)\]
\[2\xi_1 -3 \xi_2 + \xi_3 - \xi_4 , \; N(2, 21)\]
\[P(|2\xi_1 -3 \xi_2 + \xi_3 - \xi_4 | < 13) = \text{Ф}\left(\frac{13-2}{\sqrt{21}}\right) - \text{Ф}\left(\frac{-13-2}{\sqrt{21}}\right)
\]
Ответ:  \textbf{0.9918}

\end{document}
