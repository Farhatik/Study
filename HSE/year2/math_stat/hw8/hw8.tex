\documentclass[a4paper, 12pt]{article} % добавить leqno в [] для нумерации слева


%%% Работа с русским языком
\usepackage{cmap}					% поиск в PDF
\usepackage{mathtext} 				% русские буквы в фомулах
\usepackage[T2A]{fontenc}			% кодировка
\usepackage[utf8]{inputenc}			% кодировка исходного текста
\usepackage[english,russian]{babel}	% локализация и переносы

%%% Дополнительная работа с математикой
\usepackage{amsmath,amsfonts,amssymb,amsthm,mathtools} % AMS
\usepackage{icomma} % "Умная" запятая: $0,2$ --- число, $0, 2$ --- перечисление

%% Номера формулx
\mathtoolsset{showonlyrefs=true} % Показывать номера только у тех формул, на которые есть \eqref{} в тексте.

%% Шрифты
\usepackage{euscript}	 % Шрифт Евклидxwxx
\usepackage{mathrsfs} % Красивый матшрифт
\usepackage{ dsfont }
%% Свои команды
\DeclareMathOperator{\sgn}{\mathop{sgn}}
\newcommand{\Expect}{\mathsf{E}}

%% Перенос знаков в формулах (по Львовскому)
\newcommand*{\hm}[1]{#1\nobreak\discretionary{}
{\hbox{$\mathsurround=0pt #1$}}{}}

%%% Заголовок
\author{Агаев Фархат}
\title{Домашнее задание по мат стату №8}
\date{\today}

\begin{document} % конец преамбулы, начало документа

\maketitle
\section*{Задание №11}
\[\xi  - \text{ случайная велечина } \]
Нужно найти:
\[ \Expect(sin(2 \xi) | tg (\xi)) \]
Расскроем по триганометрической формуле 
\[sin(2x) = \frac{2tg(x)}{1 + tg^2(\xi)} \]
Тогда 
gолучим
\[ \Expect(\frac{2tg(x)}{1 + tg^2(\xi)} | tg (\xi)) \]
теперь Делаем замену  $tg(\xi) = \eta$ по Лемме с лекции получаем
\[\frac{2\eta}{1 + \eta^2} \cdot \Expect(1 | \eta)\] 
Далее 
\[\frac{2\eta}{1 + \eta^2} \]

Ставим обратную замену и получаем  \\\\
\textbf{Ответ:}
\[\frac{2tg(\xi)}{1 + tg^2(\xi)} \]


\end{document}