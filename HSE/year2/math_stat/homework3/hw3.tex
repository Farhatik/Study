\documentclass[a4paper, 12pt]{article} % добавить leqno в [] для нумерации слева


%%% Работа с русским языком
\usepackage{cmap}					% поиск в PDF
\usepackage{mathtext} 				% русские буквы в фомулах
\usepackage[T2A]{fontenc}			% кодировка
\usepackage[utf8]{inputenc}			% кодировка исходного текста
\usepackage[english,russian]{babel}	% локализация и переносы

%%% Дополнительная работа с математикой
\usepackage{amsmath,amsfonts,amssymb,amsthm,mathtools} % AMS
\usepackage{icomma} % "Умная" запятая: $0,2$ --- число, $0, 2$ --- перечисление

%% Номера формул
\mathtoolsset{showonlyrefs=true} % Показывать номера только у тех формул, на которые есть \eqref{} в тексте.

%% Шрифты
\usepackage{euscript}	 % Шрифт Евклид
\usepackage{mathrsfs} % Красивый матшрифт
\usepackage{ dsfont }
%% Свои команды
\DeclareMathOperator{\sgn}{\mathop{sgn}}
\newcommand{\Expect}{\mathsf{E}}

%% Перенос знаков в формулах (по Львовскому)
\newcommand*{\hm}[1]{#1\nobreak\discretionary{}
{\hbox{$\mathsurround=0pt #1$}}{}}

%%% Заголовок
\author{Агаев Фархат}
\title{Домашнее задание по мат стату №3}
\date{\today}

\begin{document} % конец преамбулы, начало документа

\maketitle
\section*{Задача №10}
(a)
 $\eta$ имеет плотность $\rho(x) = 1 - |x|$ при $|x| \leq 1$ 
и $\rho(x) = 0$ при $|x| > 1$
 \[
   \phi_{\eta}(t) = \mathds{E}e^{it\eta} = \int_{-\infty}^{+\infty}e^{itx}\rho(x)dx = \int_{-1}^{1} e^{itx}(1 - |x|)dx = -\frac{e^{-it}(-1 + e^{it})^2}{t^2}
    \]
(b) Мы знаем, что $A,U,V$ - нез. сл. вел. и соответ хар. ф-ии 
$\phi_u, \phi_v$ также $P(A = 1) = 1 - P(A = 0) = p$ 
Воспользуемся независимостью случ. вел. и раскроем по линейности хар. ф-ию
для случайной величины  $\eta = AU + (1 - A)V$.
\[
   \phi_{\eta}(t) = \mathds{E}e^{it\eta} = \mathds{E}e^{it(AU + (1 - A)V)} = 
   \mathds{E}e^{itAU}e^{it(1 - A)V}
    \]
    \[
        \mathds{E}{e^{itU}}^Ae^{itV^{(1 - A)}} = (\phi_u\phi_v)^p
    \]
\section*{Задача №11}
Ответ, очевидно, что да является существует. Распишем

\[
\phi_Y(t) = \mathds{E}costX = \mathds{E}\frac{1}{2} e^{itX} + \frac{1}{2}e^{-itx} 
= \mathds{E}e^{itX\xi}
\]
$X, \xi$ - независимые и $\xi$ принимает значение 1 или -1 с вер. $\frac{1}{2}$
 
\end{document}
