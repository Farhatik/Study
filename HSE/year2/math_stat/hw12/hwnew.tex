\documentclass[a4paper, 12pt]{article} % добавить leqno в [] для нумерации слева


%%% Работа с русским языком
\usepackage{cmap}					% поиск в PDF
\usepackage{mathtext} 				% русские буквы в фомулах
\usepackage[T2A]{fontenc}			% кодировка
\usepackage[utf8]{inputenc}			% кодировка исходного текста
\usepackage[english,russian]{babel}	% локализация и переносы

%%% Дополнительная работа с математикой
\usepackage{amsmath,amsfonts,amssymb,amsthm,mathtools} % AMS
\usepackage{icomma} % "Умная" запятая: $0,2$ --- число, $0, 2$ --- перечисление

%% Номера формулx
\mathtoolsset{showonlyrefs=true} % Показывать номера только у тех формул, на которые есть \eqref{} в тексте.

%% Шрифты
\usepackage{euscript}	 % Шрифт Евклидxwxx
\usepackage{mathrsfs} % Красивый матшрифт
\usepackage{ dsfont }
%% Свои команды
\DeclareMathOperator{\sgn}{\mathop{sgn}}
\newcommand{\Expect}{\mathsf{E}}

%% Перенос знаков в формулах (по Львовскому)
\newcommand*{\hm}[1]{#1\nobreak\discretionary{}
{\hbox{$\mathsurround=0pt #1$}}{}}

%%% Заголовок
\author{Агаев Фархат}
\title{Домашнее задание по мат стату №12}


\begin{document} % конец преамбулы, начало документа



\section*{Листок №8 Задание №10}
Дана выборка размера 1 из кспоненциального распределения с параметром 
$\lambda$. Для проверки гипотезы $H_0 : \lambda = 1$ против $H_1 : \lambda =3$
используют критерий, если $X_1 > 3$, то $H_0$ оклонить, а если $X_1 \leq 3$, то принять. 
Найдите уровень значимости и мощность критерия.
\subsection*{Уровень значимости (а)}

Просто посчитаем по определению нужную вероятность 
с помошью известной плотности экспоненциального распределения
$\rho(x) = \lambda e^{-\lambda x}$

\[
P(X \in K) (когда \; \lambda = 1) = \int_{-\infty}^{\infty} 
I_{X > 3}(x) \cdot 1 * e^{-1 * x} dx = \int_{3}^{\infty} e^{-x}dx = \frac{1}{e^{3}} \Rightarrow
\]

\[
    Уровень \; значимости \; \alpha \approx 0.0497    
\]

\subsection*{Мощность критерия (б)}
Аналогичном образом, но уже подругой формуле посчитаем мощность критерия

\[
P(X \in K) ( \; \lambda = 3) = \int_{-\infty}^{\infty} 
I_{X > 3}(x) \cdot 3 e^{-3 x} dx = \int_{3}^{\infty} 3e^{-3x}dx = \frac{1}{e^{9}}
= 1 - \beta \Rightarrow 
\]

\[
    Мощность \; критерия \; \beta \; = 1 -  \frac{1}{e^{9}} \approx 0.9998766
\]


\section*{Листок №8 Задание №8}

Критерий Неймана-Пирсона Гласит, что критическо множество $K$ 

\[
    K = \{ X : \frac{f}{g} \geq t\}   
\]
Так как мы знаем что $f$ - это плотность равная $e^{-2|x|}$ 
$g$ - это плотность нормального распределения равная $\frac{1}{\sqrt{2\pi}}e^{\frac{-x^2}{2}}$
\\§
По сути нам нужно из неравенства выразить x и позже по определению вероятности 
мы сможем посчитать $\alpha$ уровнь значимости нашего критерия.

\[
    \frac{e^{-2|x|}}{\frac{1}{\sqrt{2\pi}}e^{\frac{-x^2}{2}}} \geq t 
    \Rightarrow e^{-2|x| + \frac{x^2}{2}} \geq \frac{t}{\sqrt{2\pi}}
    \] 
\[ 
    \Rightarrow -2|x| + \frac{x^2}{2} \geq \ln(\frac{t}{\sqrt{2\pi}})
    \Rightarrow x^2 - 4|x| - 2\ln(\frac{t}{\sqrt{2\pi}}) \geq 0
\]
ДИСКРИМИНАНТ СИЛА! \\
Сделаем маленькую замену $|x| = a \Rightarrow x^2 = a^2 $
\\ Опа по формулке, мы получим:
\[
    a_{1,2}  = \frac{4 \; \pm \; \sqrt{16 + 8\ln(\frac{t}{\sqrt{2\pi}}})}{2}     
\]
\[
    a_1 =  2 \; - \; \sqrt{4 + 2\ln(\frac{t}{\sqrt{2\pi}}}) \; \; \;
    a_2 =  2 \; + \; \sqrt{4 + 2\ln(\frac{t}{\sqrt{2\pi}}})
    \]
Так, следующий этап грамотно найти значение параметра $t$, 
чтобы получить нужный уровень значимости $\alpha$
Мы знаем, что 
\[
\alpha = P_{\theta_0}(X \in K) = \int_{K}\rho_0(x)dx =  \int_{K} \frac{1}{\sqrt{2\pi}}e^{\frac{-x^2}{2}} 
\]
Грамотно раскрыв модуль на иксе, мы сможем найти гранциы для интеграла
\[
    \alpha =  \int_{a_2}^{+\infty}\frac{1}{\sqrt{2\pi}}e^{\frac{-x^2}{2}} 
    + \int_{-\infty}^{a_1}\frac{1}{\sqrt{2\pi}}e^{\frac{-x^2}{2}} + \int_{a_1}^{a_2}\frac{1}{\sqrt{2\pi}}e^{\frac{-x^2}{2}}
    \]


\end{document}
