\documentclass[a4paper, 12pt]{article} % добавить leqno в [] для нумерации слева


%%% Работа с русским языком
\usepackage{cmap}					% поиск в PDF
\usepackage{mathtext} 				% русские буквы в фомулах
\usepackage[T2A]{fontenc}			% кодировка
\usepackage[utf8]{inputenc}			% кодировка исходного текста
\usepackage[english,russian]{babel}	% локализация и переносы

%%% Дополнительная работа с математикой
\usepackage{amsmath,amsfonts,amssymb,amsthm,mathtools} % AMS
\usepackage{icomma} % "Умная" запятая: $0,2$ --- число, $0, 2$ --- перечисление

%% Номера формул
\mathtoolsset{showonlyrefs=true} % Показывать номера только у тех формул, на которые есть \eqref{} в тексте.

%% Шрифты
\usepackage{euscript}	 % Шрифт Евклид
\usepackage{mathrsfs} % Красивый матшрифт

%% Свои команды
\DeclareMathOperator{\sgn}{\mathop{sgn}}

%% Перенос знаков в формулах (по Львовскому)
\newcommand*{\hm}[1]{#1\nobreak\discretionary{}
{\hbox{$\mathsurround=0pt #1$}}{}}

%%% Заголовок
\author{Агаев Фархат}
\title{Домашнее задание по дискретной математике №3}
\date{\today}
\begin{document} % конец преамбулы, начало документа

\maketitle

\section*{Задача №1}
Всякий раз перед дождём Петя чихает. Как-то раз Петя
чихнул. «Значит будет дождь» — подумал он. Правильно ли рассуждал
Петя?
\begin{itemize}
    \item A - Дождь идет
    \item B - Петя чихнул
\end{itemize}
\[A \rightarrow B\] 
Очевидно, что из этого не следует 
\[B \rightarrow A\] 
Следовательно, неправильно.
\section*{Задача №2}
\subsection*{Ответ:}
\[(p \And q \And \bar{r}) 
\vee ( p \And \bar{q} \And q) 
\vee (\bar{p} \And q \And r) \]
Очевидно исходя из таблицы




\section*{Задача №3}
Построим таблицу истинности для
\[(u \rightarrow  v) \rightarrow (w \wedge u)\]

\begin{center}
    \begin{tabular}{ |c|c|c|c| } 
     \hline
     $u$ & $v$ & $w$ & $f$ \\ 
     \hline
     0 & 0 & 0 & 0 \\ 
     \hline
     0 & 0 & 1 & 0 \\ 
     \hline
     0 & 1 & 0 & 0 \\ 
     \hline
     0 & 1 & 1 & 0 \\ 
     \hline
     1 & 0 & 0 & 1 \\ 
     \hline
     1 & 0 & 1 & 1 \\ 
     \hline
     1 & 1 & 0 & 0 \\ 
     \hline
     1 & 1 & 1 & 1 \\ 
     \hline
    \end{tabular}
\end{center}
\textbf{КНФ}
\[(u \vee v \vee w) \wedge 
(u \vee v \vee \bar{w}) \wedge
(u \vee \bar v \vee w) \wedge
(u \vee \bar v \vee \bar w) \wedge
(\bar u \vee \bar v \vee w) \]


\section*{Задача №4}
Доказать в исчислении резолюций несовместность набора
дизъюнктов $a \vee b, \; b \vee c, \; c \vee d, \; d \vee e, \; e \vee a, \;
 \bar a \vee \bar b, \; \bar b \vee \bar c, \; \bar c \vee \bar d, \; 
 \bar d \vee \bar e, \; \bar e \vee \bar a $  \\ \\
 На лекции была \textbf{теорема} корректности исчисления резолюций 

 \textit{Если у множества дизъюнктов можно вывести пустой дизъюнкт $\Rightarrow$
это множество несовместно.}\\

\textbf{Правило резолюций}
\[
    \frac{A \vee p, \; B \vee \bar{p}}{A \vee B}
\]

\textbf{Решение}
\begin{align*}
    \frac{a \vee b, \; \bar{b} \vee \bar{c}}{a \vee \bar c} \\ \\
    \frac{a \vee \bar c, \; c \vee d}{ a \vee d} \\ \\
    \frac{a \vee d, \; \bar{d} \vee \bar{e}}{ a \vee \bar e} \\ \\
    \frac{a \vee e, \; a \vee \bar{e}}{a \vee a  = a} \\ \\
    \frac{\bar a \vee \bar b, \; b \vee c}{\bar a \vee c} \\ \\
    \frac{\bar a \vee  c, \; \bar c \vee \bar d}{ \bar a \vee \bar d} \\ \\
    \frac{\bar a \vee \bar d, \; d \vee e}{\bar a \vee  e} \\ \\
    \frac{\bar a \vee \bar e, \; \bar a \vee e}{\bar a \vee \bar a  = \bar{a}}
\end{align*}

Отсюда, очевидно $\Rightarrow$
\[
\frac{a, \; \bar{a}}{Пустой \;дизъюнкт}
\]

ЧТД
 
\section*{Задача №5}
Можно ли в исчислении резолюций из набора дизъюнктов 
$p \vee q, \; \bar p \vee q \vee r, \; p \vee \bar q \vee r, \;
\bar p \vee \bar r, \; p \vee \bar q \vee \bar r$
вывести пустой дизъюнкт?

\begin{align*}
    &\frac{p \vee \bar q \vee r, \; p \vee \bar q \vee \bar r}{p \vee \bar q} \\ \\
    &\frac{p \vee q, \;p \vee \bar q}{p} \\ \\
    &\frac{p \vee q, \;\bar p \vee q}{q} \\ \\
    &\frac{p, \; \bar{p} \vee  \bar r}{\bar r} 
\end{align*}
Если $p = 1, \; q = 1,\; r = 0 \Rightarrow$ совместен набор $\Rightarrow$ 
невозможно вывести пустой дизъюнкт.

\section*{Задача №6}
Добавим к исчислению резолюций правило, которое позволяет вывести из дизъюнкта
$A$
любой дизъюнкт вида
$A \vee B$
Расширит
ли это добавление возможности исчисления резолюций (можно 
ли будет с помощью этого правила доказать несовместность какого-нибудь
набора дизъюнктов, для которого раньше это было невозможно)?

\textbf{Ответ:} Данное правило никак не расширит возможности исчисления резолюций.
Так как если множество было несовместным, то очевидно, что мы могли получить пустой дизъюнкт.
Данный дизъюнкт мы получим ровно таким же образом. Если множество было совместным, 
то было означивание, при котором будет верно $A$ и $A \vee B$ 
(ибо если $A$ - истинный, то очев, что $A \vee B$)

\section*{Задача 7}
Вывести в исчислении высказываний формулу 
$p \rightarrow ((p \rightarrow q) \rightarrow q)$
(можно пользоваться леммой о дедукции)
\begin{enumerate}
    \item Используем Модус Понус  
    \\ $p, \; p \rightarrow q \vdash q$
    \item  Используем лемму о Дедукции и получаем.
     \\ $p \vdash (p \rightarrow q) \rightarrow q$
    \item Используем лемму о Дедукции и получаем.
     \\ $\vdash p \rightarrow 
     (p \rightarrow q) \vdash (p \rightarrow q) \rightarrow q$
\end{enumerate}
\section*{Задача 8}
a) \begin{enumerate}
    \item $\vdash ((p \rightarrow q) \rightarrow p) \rightarrow p$ \\
    \item  Используем лемму о Дедукции \\ 
    $(p \rightarrow q) \rightarrow p) \vdash p$ \\
    \item Используем правило исчерпывающего разбора случаев \\
    \\ 1. $(p \rightarrow q) \rightarrow p), \; p \vdash p$,
     2. $(p \rightarrow q) \rightarrow p), \; \bar p \vdash p$
\end{enumerate}
\[1. \;(p \rightarrow q) \rightarrow p), \; p \vdash p \; [\text{Ибо } p \vdash p] \]
\begin{align*}
    &2.\;(p \rightarrow q) \rightarrow p), \; \bar p \vdash p \\
    &Аксоима \; 9 \\ 
    &\bar{p} \rightarrow (p \rightarrow q) \\
    &Для \; \bar{p}, \; \bar{p} \rightarrow (p \rightarrow q) \; 
    \; используем \; Модус \; Понус  \\
    & Получаем \; (p \rightarrow q) \\
    &Для \; (p \rightarrow q), \; (p \rightarrow q) \rightarrow p \; 
    \; используем \; Модус \; Понус \\
    & Получаем \; p
\end{align*}
б) \begin{enumerate}
    \item Аксиома 1
    \\ $p \vee \bar p$
    \item Аксиома 2 \\
    $(p \rightarrow q) \rightarrow (p \rightarrow q) \vee p$
    \item Аксиома 3 \\
    $p \rightarrow (p \rightarrow q) \vee p$
    \item Аксиома 4 \\
    $\bar p \rightarrow (p \rightarrow q)$
    \item Аксиома 5 \\
    $(p \rightarrow (p \rightarrow q) \vee p) \rightarrow 
    ((\bar p \rightarrow (p \rightarrow q) \vee p) \rightarrow((p \vee \bar p) 
    \rightarrow (p \rightarrow q) \vee p))$
    \item Используем правило сечения для 2 и 4 \\
    Получаем $\bar p \rightarrow (p \rightarrow q) \vee p$
    \item Используем Модус Понус для 3-ей строчки и 5-ой\\
    $(\bar p \rightarrow (p \rightarrow q) \vee p) \rightarrow((p \vee \bar p) 
    \rightarrow (p \rightarrow q) \vee p)$
    \item Используем Модус Понус 6-ой и 7-ой\\
     $(p \vee \bar p) 
    \rightarrow (p \rightarrow q) \vee p$
    \item Используем Модус Понус для 1-ой  и 8-ой и получаем \textbf{ответ}\\
    $(p \rightarrow q) \vee p$
\end{enumerate}


\end{document}

