\documentclass[a4paper, 12pt]{article} % добавить leqno в [] для нумерации слева


%%% Работа с русским языком
\usepackage{cmap}					% поиск в PDF
\usepackage{mathtext} 				% русские буквы в фомулах
\usepackage[T2A]{fontenc}			% кодировка
\usepackage[utf8]{inputenc}			% кодировка исходного текста
\usepackage[english,russian]{babel}	% локализация и переносы

%%% Дополнительная работа с математикой
\usepackage{amsmath,amsfonts,amssymb,amsthm,mathtools} % AMS
\usepackage{icomma} % "Умная" запятая: $0,2$ --- число, $0, 2$ --- перечисление

%% Номера формул
\mathtoolsset{showonlyrefs=true} % Показывать номера только у тех формул, на которые есть \eqref{} в тексте.

%% Шрифты
\usepackage{euscript}	 % Шрифт Евклид
\usepackage{mathrsfs} % Красивый матшрифт

%% Свои команды
\DeclareMathOperator{\sgn}{\mathop{sgn}}

%% Перенос знаков в формулах (по Львовскому)
\newcommand*{\hm}[1]{#1\nobreak\discretionary{}
{\hbox{$\mathsurround=0pt #1$}}{}}

%%% Заголовок
\author{Агаев Фархат}
\title{Домашнее задание по теор вероятности  №8}
\date{\today}

\begin{document} % конец преамбулы, начало документа

\maketitle
\section*{Задача  №9}
Начальное распределение $\mu = (0, 0, 0, 0, 0, 1)$  \\
Мы знаем, что при устремление $k \rightarrow \infty $ \[
    M^k = \mu \cdot P^k \rightarrow M  -  \text{ стационарное распределение}
\] 
Также мы знаем, что \begin{align*}
    &M = M P \\ 
    &M = (a, b, c, e, d, f) \\ 
    &a + b + c + e + d + f= 1 \\ 
    &P = \begin{pmatrix}
        0 & 0.25 & 0.25 & 0 & 0.25 & 0.25\\
        0.25 & 0& 0 & 0.25 & 0.25 & 0.25 \\
        0.25 & 0& 0 & 0.25 & 0.25 & 0.25 \\
        0 & 0.25 & 0.25 & 0 & 0.25 & 0.25\\
        0.25 & 0.25 & 0.25 & 0.25 & 0 & 0\\
        0.25 & 0.25 & 0.25 & 0.25 & 0 & 0\\
    \end{pmatrix}
\end{align*}
Составим линейное уравнение и решим систему. \\
Посчитаем с помощью вольфрам альфа.  \\
$M = \frac{1}{6}(1, 1, 1, 1, 1, 1 ) \Rightarrow$ \\
Ответ: 1/6 


\section*{Задача  №11}
$ P = \begin{pmatrix}
    0.1 & 0.5 & 0.4\\
    0.6 & 0.2 & 0.2 \\
    0.3 & 0.4 & 0.3
\end{pmatrix}
\; \mu = (0.7, 0.2, 0.1)
$ \\
Воспользуемся формулой 
\[M^k = \mu \cdot P^k
    \]
Это как раз, то что мы ищем при k = 2
\[M^2 = (0.385, 0.366, 0.279 ) = 
(P(\xi_2 = 1), P(\xi_2 = 2), P(\xi_2 = 3))
\]
Найдем стационарное с помощью 
\begin{align*}
    M = M P \\ 
    M = (a, b, c) \\ 
    a + b + c = 1
\end{align*}
Составим линеное уравнение и решим его (вольфрам альфа)
получаем, что \[M = \frac{1}{17}(17,16,14)\]



\end{document}
