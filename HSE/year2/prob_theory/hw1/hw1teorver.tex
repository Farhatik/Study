\documentclass[a4paper, 12pt]{article} % добавить leqno в [] для нумерации слева


%%% Работа с русским языком
\usepackage{cmap}					% поиск в PDF
\usepackage{mathtext} 				% русские буквы в фомулах
\usepackage[T2A]{fontenc}			% кодировка
\usepackage[utf8]{inputenc}			% кодировка исходного текста
\usepackage[english,russian]{babel}	% локализация и переносы

%%% Дополнительная работа с математикой
\usepackage{amsmath,amsfonts,amssymb,amsthm,mathtools} % AMS
\usepackage{icomma} % "Умная" запятая: $0,2$ --- число, $0, 2$ --- перечисление

%% Номера формул
\mathtoolsset{showonlyrefs=true} % Показывать номера только у тех формул, на которые есть \eqref{} в тексте.

%% Шрифты
\usepackage{euscript}	 % Шрифт Евклид
\usepackage{mathrsfs} % Красивый матшрифт

%% Свои команды
\DeclareMathOperator{\sgn}{\mathop{sgn}}

%% Перенос знаков в формулах (по Львовскому)
\newcommand*{\hm}[1]{#1\nobreak\discretionary{}
{\hbox{$\mathsurround=0pt #1$}}{}}

%%% Заголовок
\author{Агаев Фархат}
\title{Домашнее задание по теории вероятности}
\date{\today}

\begin{document} % конец преамбулы, начало документа

\maketitle

\section*{Задача №12}

\paragraph{Обозначим:}
\begin{align}
    &A_{1} - \text{событие того, что люди НЕ вышли на первом этаже}\\
    &A_{2} - \text{событие того, что люди НЕ вышли на втором этаже}\\
    &A_{3} - \text{событие того, что люди НЕ вышли на третьем этаже}\\
    &A_{4} - \text{событие того, что люди НЕ вышли на четвертом этаже}\\ 
\end{align}
Теперь мы данную задачу будем решать использую объединение и дополнение, а именно:

Пусть $A_{1} \cup A_{2} \cup A_{3} \cup A_{4}$ - это событие,
 когда люди не вышли на первом, 
или на втором, или на третьем,
 или на четвертом этаже дополнние к этому событию будет
  $\overline{A_{1} \cup A_{2} \cup A_{3} \cup A_{4}}
  = \bar A_{1} \cap \bar A_{2} \cap \bar A_{3} \cap \bar A_{4}$ - это и есть
  то, что мы ищем, то есть чтобы на каждом этаже вышел человек.

  Дальше мы просто посчитаем вероятность события $A_{1} \cup A_{2} \cup A_{3} \cup A_{4}$ 
  с помощью формулы включений-исключений, после вероятность дополнения
  по формуле [1 - Pr($A_{1} \cup A_{2} \cup A_{3} \cup A_{4}$)] таким образом:
  \begin{align}
    \text{Pr} (A_{1} \cup A_{2} \cup A_{3} \cup A_{4}) = \sum\limits_{i=1}^4 Pr(A_{i})
    - \sum\limits_{1 \leq i < j \leq 4} Pr(A_{i} \cap A_{j}) + \\
    + \sum\limits_{1 \leq i < j < k \leq 4} Pr(A_{i} \cap A_{j} \cap A_{k}) + 0 
\end{align}
    
0 - это это когда все четыре события выполнены, а это не возможно, так как  по условие известно,
что каждый человек вышел на каком-либо этаже $\rightarrow$ невозможно случая когда на каждом этаже не вышли люди

Очевидно, что $A_{1} = A_{2} = A_{3} = A_{4} = {(\frac{3}{4})}^{10}$ так как каждый человек может выйти на трех этажах, кроме своего
(ледей всего 10),
 отсюда и получаем дробь.

 Таким же образом $Pr(A_{i} \cap A_{j}) = {(\frac{2}{4})}^{10}$, $Pr(A_{i} \cap A_{j} \cap A_{k}) = {(\frac{1}{4})}^{10}$
дальше пользуемся формулой сочетаний и получаем:
\begin{align} 
     4 \cdot Pr(A_{i}) - {2 \choose 4}Pr(A_{i} \cap A_{j}) + {3 \choose 4}Pr(A_{i} \cap A_{j} \cap A_{k}) \approx 0,225 \Rightarrow \\
     \text{Получаем ответ 1 - 0,225 = 0,775}
\end{align}

\section*{Задача №11}
\paragraph*{a)}
максимально очевидно, что мы посчитаем с помощью формулы число сочетаний (то есть просто выберем 6 математиков 
из 8 и 3 физиков из 12, всего нужно выбрать 9 из 20 человек):
\[
    \frac{{6 \choose 8} \cdot {3 \choose 12}}{{9 \choose 20}}
\]
\paragraph*{b)}
Меньше трех физиков это либо 2-ое $\Rightarrow \frac{{7 \choose 8} \cdot {2 \choose 12}}{{9 \choose 20}}$

либо 1 физик $\Rightarrow \frac{1 \cdot {1 \choose 12}}{{9 \choose 20}}$, возьем просто сумму и получим ответ:

\[
    \frac{{7 \choose 8} \cdot {2 \choose 12}}{{9 \choose 20}} + \frac{1 \cdot {1 \choose 12}}{{9 \choose 20}}
\]

\section*{Задача №10}

Всего 32 карты, 4 туза, тогда пусть нашим вероятностным пространством
будет пары карт для прикупа $\Rightarrow$ всего пар $32 \cdot 31$, а пар для тузов $4 \cdot 3$ отсюда мы можем сразу посчитать веротность для пунтка:
\paragraph*{a)}
\[ \frac{4 \cdot 3}{32 \cdot 31}
    \]

    \paragraph*{b)}
    Если нам известно, что у одного игрока нет тузов, а у него 10 карт $\Rightarrow$ что у нас останется 22 карты $\Rightarrow$

\[ \frac{4 \cdot 3}{22 \cdot 21}
    \]

\section*{Задача №9}
Пусть наше вероятностное пространство будет перетсановка из 20 чисел,
очевдно, что всего таких перестановок 20!, нам подходящие перестановки, это
когда на первом месте может оказаться один из десяти мальчиков, на втором одна
их десяти девочек, на третьем один из девяти мальчик и так далее чередуюсь
то есть всего (если также учесть, что можно начать наше чередование с девочки)
будет $10! \cdot 10! \cdot 2$ и тогда ответ:
\[\frac{10! \cdot 10! \cdot 2}{20!}
    \]

\end{document}
