\documentclass[a4paper, 12pt]{article} % добавить leqno в [] для нумерации слева


%%% Работа с русским языком
\usepackage{cmap}					% поиск в PDF
\usepackage{mathtext} 				% русские буквы в фомулах
\usepackage[T2A]{fontenc}			% кодировка
\usepackage[utf8]{inputenc}			% кодировка исходного текста
\usepackage[english,russian]{babel}	% локализация и переносы

%%% Дополнительная работа с математикой
\usepackage{amsmath,amsfonts,amssymb,amsthm,mathtools} % AMS
\usepackage{icomma} % "Умная" запятая: $0,2$ --- число, $0, 2$ --- перечисление

%% Номера формул
\mathtoolsset{showonlyrefs=true} % Показывать номера только у тех формул, на которые есть \eqref{} в тексте.

%% Шрифты
\usepackage{euscript}	 % Шрифт Евклид
\usepackage{mathrsfs} % Красивый матшрифт

%% Свои команды
\DeclareMathOperator{\sgn}{\mathop{sgn}}

%% Перенос знаков в формулах (по Львовскому)
\newcommand*{\hm}[1]{#1\nobreak\discretionary{}
{\hbox{$\mathsurround=0pt #1$}}{}}

%%% Заголовок
\author{Агаев Фархат}
\title{Домашнее задание по теор вероятности  №5}
\date{\today}

\begin{document} % конец преамбулы, начало документа

\maketitle

\subsection*{Задача №10}
\begin{align*} 
    &N = 2000  \text{ - количество изделий} \\
    &p_{n} = 0,001 -\text{ вероятность того, что изделие бракованное }\\
    &p = p_n * (1 - 0,9) = 0,0001 \text{ - вероятность того, } \\
    &\text{что изделие бракованное и тест был отрицательный} \\ 
    &\text{(прошло тест незамеченным)} \\
    &q = 1 - 0,0001 = 0,9999 
\end{align*}

Вероятностное пространство состоит из наборов 
изделий размера N. 

Каждый набор состоит из изделий двух типов.

1) Изделие - бракованное и тест отриц (незамеченный).

2) Изделие - небракованное или тест полож (заметили).

Тогда, очевидно, что 
\[P(\text{к бракованных и незамеченных}) = C^k_N\cdot p^k \cdot q ^{N - k}\]
\[\text{Ответ: } C^{k}_{2000} \cdot 0.0001^k \cdot 0.9999^{2000 - k} \]
\end{document}
