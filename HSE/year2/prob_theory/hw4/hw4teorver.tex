\documentclass[a4paper, 12pt]{article} % добавить leqno в [] для нумерации слева


%%% Работа с русским языком
\usepackage{cmap}					% поиск в PDF
\usepackage{mathtext} 				% русские буквы в фомулах
\usepackage[T2A]{fontenc}			% кодировка
\usepackage[utf8]{inputenc}			% кодировка исходного текста
\usepackage[english,russian]{babel}	% локализация и переносы

%%% Дополнительная работа с математикой
\usepackage{amsmath,amsfonts,amssymb,amsthm,mathtools} % AMS
\usepackage{icomma} % "Умная" запятая: $0,2$ --- число, $0, 2$ --- перечисление

%% Номера формул
\mathtoolsset{showonlyrefs=true} % Показывать номера только у тех формул, на которые есть \eqref{} в тексте.

%% Шрифты
\usepackage{euscript}	 % Шрифт Евклид
\usepackage{mathrsfs} % Красивый матшрифт

%% Свои команды
\DeclareMathOperator{\sgn}{\mathop{sgn}}

%% Перенос знаков в формулах (по Львовскому)
\newcommand*{\hm}[1]{#1\nobreak\discretionary{}
{\hbox{$\mathsurround=0pt #1$}}{}}

%%% Заголовок
\author{Агаев Фархат}
\title{Домашнее задание по теории вероятности №4}
\date{\today}

\begin{document} % конец преамбулы, начало документа

\maketitle

\section*{Задача № 10}

\paragraph{Обозначим:}
\begin{itemize}
    \item $A_i$ - событие, что болен i-ой болезнью.
    \item $p_i$ - вероятность, что болен i-ой болезнью.
    \item $p_1 = \frac{1}{2}, \quad p_2 = \frac{1}{6}, \quad p_3 = \frac{1}{3}$
    \item $B$ - событие, что тест дал 
    4 положительных и 1 отрицательный результат.
\end{itemize}


Посчитаем по формуе полной вероятности:
\begin{align*}
P(B_{}) = C^4_5 \cdot (&P(B_{}|A_1) \cdot P(A_1)  \; + \\
+ \; &P(B_{}|A_2) \times P(A_2) \;+  P(B_{}|A_3)) \cdot P(A_3)) = \\
&5 \cdot (0,1^4 \cdot 0,9 \cdot 0,5  + 0,2^4 \cdot 0,8 \cdot (1/6) + 0,9^4 \cdot 0,1 \cdot (1/3) \approx 0,1105
 \end{align*}

\subsubsection*{Ответ} 

По формуле Байеса:

\begin{equation*}
    P(A_1|B_{})= \frac{P(B_{}|A_1)\cdot P(A_1)}{P(B_{})} = \frac{0,1^4 \cdot 0,9 \cdot 0,5 \cdot 5}{0,1105} \approx 0,002
\end{equation*}

\begin{equation*}
    P(A_2|B_{})= \frac{P(B_{}|A_2)\cdot P(A_2)}{P(B_{})} = \frac{0,2^4 \cdot 0,8 \cdot 5 \cdot (1/6)}{0,1105} \approx 0,0009
\end{equation*}

\begin{equation*}
    P(A_3|B)= \frac{P(B|A_3)\cdot P(A_3)}{P(B)} = \frac{09^4 \cdot 0,1 \cdot (1/3) \cdot 5}{0,1105} \approx 0,9886 
\end{equation*}

\section*{Задача №11}
\paragraph{Обозначим:}
\begin{itemize}
    \item $A^2_N, $ - событие, что при N испытаниях будет ровно 2 успеха.
    \item $A^{2k}_{N}$ - событие, что при N испытаниях будет четное число успехов.
    \item $p = \frac{1}{2}$ - вероятность успеха.
    \item $q = \frac{1}{2}$ - вероятность провала.
\end{itemize}

\begin{equation*}
    P(A^2_N) = C^2_N \cdot \left( \frac{1}{2} \right)^N
\end{equation*}

\begin{align*}
    P(A^{2k}_N) = &\frac{C^0_N + C^2_N +C^4_N +C^6_N + \ldots}{2^N} = \\
    = &\frac{C^0_{N-1} + C^1_{N-1} +C^2_{N - 1} +C^3_{N - 1} + \ldots}{2^N} = \frac{2^{N - 1}}{2^N} 
    =\frac{1}{2}
\end{align*}

\subsection*{Ответ:}
\begin{equation*}
    P(A^2_N | A^{2k}_N) = \frac{P(A^2_N \cap A^{2k}_N)}{P(A^{2k}_N)} = 
    C^2_N \cdot \left( \frac{1}{2} \right)^{N - 1}
\end{equation*}

\section*{Задача №12}
\paragraph{Обозначим:}
\begin{itemize}
    \item $A^m_n, $ - событие, что m успехов произойдут раньше, чем n неудач.
    \item $p$ - вероятность успеха.
    \item $(1 - p)$ - вероятность неудачи
    \item кодировка, 0 - неудача, 1 - успех.
\end{itemize}
Если кодировать наши слачаи словами, то получится, что такое
\begin{align*}
    &\underbrace{.............}_{m + n - 2}1 \\
    &\underbrace{...........}_{m + n - 3}1 \\
    &\underbrace{.........}_{m + n - 4}1 \\
    &\, .........\\
    &\underbrace{11111}_{m}\\
\end{align*}
первые (m - 1 + n - 1) элементы
это m - 1 единички и n - 1 нолики,
то есть мы просто посчитаем число расстановок $C^{m-1}_{m + n - 2}$, 
таким образом посчитаем другие слова тоже $C^{m-1}_{m + n - 3}$, 
$C^{m-1}_{m + n - 4}$, \ldots, 1 \\
или если посмотреть с другой стороны по увелечению, то получим \\
$C^{m - 1}_{m - 1}, \; C^{m - 1}_{m - 2},\; \ldots,\; C^{m-1}_{m + n - 2}$

\subsection*{ответ:}
\begin{equation*}
    P(A^m_n) = \sum^{n -1}_{i = 0} C^{m - 1}_{m - 1 + i} \cdot (1 - p)^i \cdot  p^m
\end{equation*}

\end{document}