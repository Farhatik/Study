\documentclass[a4paper,12pt]{article} % добавить leqno в [] для нумерации слева

%%% Работа с русским языком
\usepackage{cmap}					% поиск в PDF
\usepackage{mathtext} 				% русские буквы в фомулах
\usepackage[T2A]{fontenc}			% кодировка
\usepackage[utf8]{inputenc}			% кодировка исходного текста
\usepackage[english,russian]{babel}	% локализация и переносы

%%% Дополнительная работа с математикой
\usepackage{amsmath,amsfonts,amssymb,amsthm,mathtools} % AMS
\usepackage{icomma} % "Умная" запятая: $0,2$ --- число, $0, 2$ --- перечисление

%% Номера формул
\mathtoolsset{showonlyrefs=true} % Показывать номера только у тех формул, на которые есть \eqref{} в тексте.

%% Шрифты
\usepackage{euscript}	 % Шрифт Евклид
\usepackage{mathrsfs} % Красивый матшрифт
\usepackage{graphicx}
\usepackage{amsfonts}

%% Перенос знаков в формулах (по Львовскому)
\newcommand*{\hm}[1]{#1\nobreak\discretionary{}
{\hbox{$\mathsurround=0pt #1$}}{}}

%%% Заголовок
\author{Фархат Агаев}
\title{Домашнее задание по дискретной математике №4}
\date{\today}

\begin{document} % конец преамбулы, начало документа

\maketitle


\section*{Задача №1} 
\textbf{Сигнатура:}
\begin{align*}
    &M(x) - \text{быть мужчиной} \\
    &F(x) - \text{быть женщиной} \\
    &C(x, y) - \text{$x$ и $y$ состоят в браке} \\
    &P(x, y) - \text{$x$ родитель $y$}
\end{align*}
\textbf{\textit{Отношение 
"$x$ брат $y$"}}

$x$ - парень, есть общие родители с $y$ и не равны.
\[B(x, y) = M(x) \wedge \exists z (P(z, y) \wedge P(z, x))
 \wedge \neg(x = y)\]
\textbf{\textit{Отношение 
"$x$ является тёщей $y$"}}

$x$ - женщина, мать жены,
$y$ - мужчина, есть жена. 
\[
    D(x, y) = M(y) \wedge
    \exists z (F(z) \wedge C(z, y) \wedge P(x, z))
    \wedge F(x) 
\]
\textbf{\textit{Отношение 
"$x$ является племянником $y$"}}

$x$ - сын брата или сестры

Выразим аналогично вначале \textbf{\textit{отношение "$a$ сестра $b$"}}
\[S(a, b) = F(a) \wedge \exists с(P(с, a) \wedge P(с, b))
 \wedge \neg(a = b)\]
 
 Отношение "$x$ является племянником $y$"
\[L(x, y) = M(x) \wedge
 \exists z(\,(B(z, y) \vee S(z, y)) \wedge P(z, x)) 
 \wedge \neg (x = y) \]
\textbf{\textit{Отношение 
"$x$ внук $y$"}}

у $x$ есть родитель $z$, он (или она) ребенок $y$.
\[K(x, y) = M(x)  \wedge  \exists z(P(z,x) \wedge (P(y,z)) \wedge \neg (x = y) \]


\section*{Задача №2} 
\textbf{Сигнатура:}
\[C(x, y) - x \text{ обманул } y\]
Запишем формулой каждый кого-то обманул:
\[\forall x_1 \, \exists y_1 \, (C(x_1, y_1))\]
Запишем формулой каждый кем-то обманут:
\[\forall  x_2\, \exists  y_2 \, (C(y_2, x_2))\]
Запишем формулой нет того, кто обманул всех:
\[\neg (\exists  x_3\, \forall  y_3 \, 
(C(x_3, y_3))) = \forall x_3 \exists y_3 \,(\neg C(x_3, y_3))\]
\textbf{Ответ:}
\[\forall x_1 \, \exists y_1 \, (C(x_1, y_1)) \wedge 
\forall  x_2\, \exists  y_2 \, (C(y_2, x_2)) \wedge
\forall x_3 \exists y_3 \,(\neg C(x_3, y_3))
    \]
    
\section*{Задача №3} 
\textbf{Сигнатура:}

\[
    \mathbb {R}, 0, +, \times, <, = 
\]
Очевидно, что нам нужно несколько условий для того, чтобы многочлен
$ax^3 + bx^2 + bx + c$ имеел корень
\begin{enumerate}
    \item $a \neq 0$, чтобы степень многочлена была равна 3.
    \item при подстановки x, многочлен должен быть равен 0.ъ
\end{enumerate} 
\textbf{Ответ:}
\[\forall a \forall b \forall c \forall \exists x 
(\neg (a = 0) \wedge
 (a \times x \times x \times x  + b \times x \times x + c \times x + d = 0)
 )\]
\section*{Задача №4}
\textbf{Сигнатура:}
\[
    \mathbb {N}, +, \times, = 
    \]
Выразим предикат быть делителем:
\[
    D(d, x) = \exists y(x = dy) 
\]
Тогда просто выразить предикат $x =$ НОД$(y,z)$
\begin{align*}
  B(x, y, z) = D(x, y) \wedge D(x, z) \wedge \neg(\exists a(k = x + a
   \wedge D(k, y) \wedge D(k, z) )  
\end{align*}
\section*{Задача №5} 
    $[\forall x (P(x) \rightarrow Q(f(x)) 
    \wedge \forall x (Q(x) \rightarrow P(f(x))
    )
    \wedge \forall x f(f(x)) = f(x)] \rightarrow 
    [
        \exists x (P(x) \vee Q(x) \rightarrow  \exists x (P(x) \wedge Q(x))
    ]
    $
    \\
    Рассмотрим случай когда данная импликация может быть равна нулю.
    Первая часть $[\forall x (P(x) \rightarrow Q(f(x)) 
    \wedge \forall x (Q(x) \rightarrow P(f(x))
    )
    \wedge \forall x f(f(x)) = f(x)] = 1$, а вторая $ [
        \exists x (P(x) \vee Q(x) \rightarrow  \exists x (P(x) \wedge Q(x))
    ] = 0$. Тогда рассмотрим дальше вторую часть откуда следует, что 
   $ \exists x (P(x) \vee Q(x)) = 1$, а $\exists x (P(x) \wedge Q(x)) = 0$, тогда должно быть так 
   $P(x) = 0$, $Q(x) = 1$ или   $P(x) = 0$, $Q(x) = 1$
   Попробуем разобраться с первой частью если $P(x) = 1$, $Q(x) = 0$.
   Все конъюкты должы быть истинными. из второго конъюкта 
   $\forall x (Q(x) \rightarrow P(f(x))$,
    очев, что  P(f(x)) должна быть истина, теперь хитрый трюк, подставляем 
    переменную $f(x)$ в первый конъюкт получаем 
    $\forall f(x) (P(f(x)) \rightarrow Q(f(f(x)))$ пользуемся третьим конъктом
    получаем $\forall f(x) (P(f(x)) \rightarrow Q(f(x))) \Rightarrow$ $\forall x Q(f(x)) = 1$,
    противоречие. Те же самые рассуждения для второго случая.
\section*{Задача №6}
\[
    \forall x \, g(f(x)) = x \wedge \exists y \forall x \neg f(x) = y
\]

Да является, пусть  $M = \mathbb{N}$, определим ф-ию $f(x) = 10x$ для всех $x$.
ф-ию $g(x) = \frac{x} {10}$ только для х кратных 10, для всех остальных x, ф-ия g(x)= 12345.
Тогда очевидно, что условие выполняется для 
\[
    \forall x \, g(f(x)) = x
\]
Также например $y = 123$ - число не кратное 10.
Мы с помощью ф-ию $f$ не сможем получить такое число $\Rightarrow$
выполняется вторая часть.

\[
    \exists y \forall x \neg f(x) = y
\]


\section*{Задача №7}
Теория называется совместной, если существует интерпритация в которой все формулы истины \\
\textbf{a)}
Первая формула $\exists x \forall y \neg P(x, y)$, существует такой х, что в отношение предиката P с любым
элементом будет всегда ответ будет ложным. Пусть этот $x = a$, тогда во второй формуле. 
$\exists y \forall x P(x, y)$ подставив  $P(y, a)$ должен быть истинным, но по первой формуле он ложный 
$\Rightarrow$ противоречие. \\
\textbf{b)}
\textbf{Ответ: да.}
Если перевести первую формулу $\forall x \neg P(x, x)$ означает антирефлексивность.
Вторая $\forall x \forall y \forall z (P(x, y) \wedge P(y, z) \rightarrow) P(x, z)$ означает транзитивность.
Последняя формула сущетсвует пара такая, что 
$\exists x \exists y (P(x, y) \wedge P(y, x))$
\\ Приведу интерпритацию. 

Пусть $P(x, y)$ = взаимно ли просты числа $x$ и $y$.

$M = \{ \mathbb{N}$ без нуля и единицы\}. Отсюда сразу все очевидно.
Натуральное число не взаимно просто с самим с собой. Вторая формула тоже выполняется. 
Третья вообще для любых (пусть $x = 2$ и $y = 3$) \\
\textbf{с)}
\textbf{Ответ: да.}
Первая формула $\forall x P(x, x)$ означает рефлексивность.
Вторая $\forall x \forall y \forall z (P(x, y) \wedge P(y, z) \rightarrow) P(x, z)$ означает транзитивность.
Третья формула, что у каждого x есть пара $\forall x \exists y P(x, y)$.
Последняя формула
$\exists x \exists y \neg(P(x, y) \wedge P(y, x))$
\\ Приведу интерпритацию.

 P(x, y) = кратность.
Очевидно выполняется первая и вторая формула, третья также выполняется если y = x, 
чтобы выполнилась последняя формула возьмем пару ($x = 4$ и $y = 2$).
\section*{Задача №8}
Для доказательства воспользуемся таким утверждением. \\
Пусть $T = \{U_1, U_2, \ldots, U_n\}$ - теория. U - формула \\ 
\[T \models A \ \Leftrightarrow \; \vdash (U_1 \wedge ...  \wedge U_n) \rightarrow U\]
\textbf{a)}
\[
   \models (\forall x Q(x) \wedge \forall(Q(x) \rightarrow P(x)) \rightarrow \forall x P(x)
\]
Очевидно, что если $Q(x) = 0$ то из лжи следует что угодно и получается истина.
Пусть $Q(x) = 1$, тогда во втором конъюнкте $P(x)$ должен быть равен 1, иначе получаем ту же 
ситуцию, а так как $P(x) = 1$, то очевидно истинность всей формулы. \\ 
\textbf{b)}
\[
   \models (\exists x Q(x) \wedge \forall x(Q(x) \rightarrow P(x)) \rightarrow \exists x P(x)
\]
Аналогично с пунктом а. нули не рассматриваем. $\forall x(Q(x) \rightarrow P(x) \Rightarrow P(x)$ тоже единица.
Рассуждения те же самые.\\
\textbf{c)}
\[
   \models (\exists x Q(x) \wedge \forall x(P(x) \rightarrow Q(x)) \rightarrow \exists x P(x)
\]
\textbf{Ответ: нет}. Пусть не сущетсвует такого $P(x)$, данное условие не будет 
противоречить импликации, также $Q(x) = 1$, то есть $\exists xP(x) ложно$, а $\forall x Q(x)$ истина, 
в таком случае импликация тоже истина, а все выражение ложно.  
\\\textbf{d)}

\[
   \models (\forall x Q(x) \wedge \forall(P(x) \rightarrow Q(x)) \rightarrow \forall x P(x)
\]
Очевидно, что нет. Аналогчно рассмотрим лишь крайний случай. Пусть $Q(x) = 1$, тогда в 
заключение будет 1, следовательно  посылка может быть отрицательной $(P(x) = 0)$ и тогда получаем, что в данном случае
формула может выдать 0. (1 $\rightarrow$ 0)
\end{document}
