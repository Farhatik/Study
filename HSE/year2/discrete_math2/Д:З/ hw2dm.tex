\documentclass[a4paper,12pt]{article} % добавить leqno в [] для нумерации слева

%%% Работа с русским языком
\usepackage{cmap}					% поиск в PDF
\usepackage{mathtext} 				% русские буквы в фомулах
\usepackage[T2A]{fontenc}			% кодировка
\usepackage[utf8]{inputenc}			% кодировка исходного текста
\usepackage[english,russian]{babel}	% локализация и переносы

%%% Дополнительная работа с математикой
\usepackage{amsmath,amsfonts,amssymb,amsthm,mathtools} % AMS
\usepackage{icomma} % "Умная" запятая: $0,2$ --- число, $0, 2$ --- перечисление

%% Номера формул
\mathtoolsset{showonlyrefs=true} % Показывать номера только у тех формул, на которые есть \eqref{} в тексте.

%% Шрифты
\usepackage{euscript}	 % Шрифт Евклид
\usepackage{mathrsfs} % Красивый матшрифт
\usepackage{graphicx}


%% Перенос знаков в формулах (по Львовскому)
\newcommand*{\hm}[1]{#1\nobreak\discretionary{}
{\hbox{$\mathsurround=0pt #1$}}{}}

%%% Заголовок
\author{Фархат Агаев}
\title{Домашнее задание по дискретной математике №2}
\date{\today}

\begin{document} % конец преамбулы, начало документа

\maketitle
\newpage
\section*{Задача №1}
Для начала докажем, что данная ф-ия вычислима:
\begin{equation*}
    \phi_{(m,n,k)}(x) = 
    \begin{cases}
        \phi_n(x) & \text{если } \phi_m(x) = 0 \\
        \phi_k(x) & \text{если } \phi_m(x) > 0 \\
        \text{не определено}& \text{если } \phi_m(x) \text{ не определено} \\
    \end{cases}
\end{equation*}

Так как ф-ии $\phi_n(x), \; \phi_m(x) \;, \phi_k(x)$ - вычислимы, то очевидно,
что и наша ф-ия $\phi_{(m,n,k)}(x)$ также вычислима (ибо мы вначале просто считаем $\phi_m(x)$ и дальше если нужно 
считаем другие вычичслимые ф-ии). 

Пусть у нас будет ф-ия которая переводит из пары натуральных чисел в натуральное число
\[ code_2(m,n) = 2^m \cdot (2n + 1) - 1\]
очевидно, что данная ф-ия - биекция, так как мы можем однозначно получить обратно два числа $m, n$ \\
Представим натуральное число $x$ в виде двоичной формы (сразу увидим нужные нам ф-ии)
\[ x  + 1 = \underbrace{10....01}_{2n + 1} \underbrace{000...0}_m\]
Теперь довольно просто получить \[code_3(m, n, k) = code_2(code_2(m, n), k)\]

% Теперь воспользуемся свойством главности ф-ии и получим то, что мы искали
\noindent так как $\{\phi_i\}$ - главная $\Rightarrow$ для любой выч. час. ф-ии $V(a, x)$ 
существует тотальная выч ф-ия $s: N \rightarrow N$ такая, что 
\[ \phi_{s(a)}(x) = V(a, x)\]

Пусть $V(a, x) = V(code_3(m, n, k), x) = \phi_{s(code_3(m,n,k))}(x)$ и тут сразу становится понятно, 
\subsection*{Ответ} 
\begin{align*}
    h(m, n, k) = s(code_3(m, n, k))
\end{align*}

\newpage
\section*{Задача №2}

Пусть 
\begin{equation*}
    f(x) = 
    \begin{cases}
       100 &\text{если } \phi_x(x) = 59\\
       59  & \text{иначе } \\
    \end{cases}
\end{equation*}
вычислима, тогда $\exists k$, $\phi_k(x) = f(x)$ посмотрим на значение ф-ии при 
x = k, тогда 
\begin{equation*}
    \phi_k(k) = 
    \begin{cases}
       100 &\text{если } \phi_k(k) = 59\\
       59  & \text{иначе } \\
    \end{cases}
\end{equation*}
Такое невозможно.


 
\section*{Задача №3}
Мы знаем, что существует перечислимое неразрешимое множество.
Возьмем 
\begin{equation*}
    X = \{x \;| \;\phi_x(x) определено\}  
\end{equation*}

\noindent$X$ - перечислимое $\Rightarrow$ существует алгоритм перечисления $P$\\
Запускаем P и он выдает какие-то $x_i$, которые мы будем присваивать нашей ф-ии $f(i) = x_i$
И тут сразу получаем искомую тот. ф-ию 
\[f:N \rightarrow N\]
очев, что $Dom(f) = N$ - разрешимое множество, $Range(f) = X$  - неразрешимое перечислимое множество.


\section*{Задача №4}
\begin{align*}
    &A = \{i \; | \; \phi_i(0) \; не \; определено\} \\
    &B = \{i \; | \; \phi_i(0) = 1\}
\end{align*}

 
% Мы знаем, что $X = \{x \;| \;\phi_x(x) \;определено\}$ - перечислимое неразрешимое множество.
% Пусть множесто А разрешимо.


% тогда 
% \[V(n, x)\ = 
% \begin{cases}
%     2000 \quad &n \in A \\ 
%     \text{не опред} &n \notin A
% \end{cases} \] 
% Очевидно, что ф-ия $V $- вычислимa
% \[V(n, x) \simeq 2000 \cdot \omega_A(x)\]
% Следовательно существует выч. тот ф-ия $S:N \rightarrow N$ такая, что \\  
% \[\forall n, x: \phi_{s(n)}(x) \simeq V(n, x)\]
% Рассмотрим случай когда $x = s(n)$/ 
% \begin{align*}
%     &\forall n: \phi_{s(n)}(s(n)) \simeq V(n, s(n)) \\
%     &s(n) \in X \Leftrightarrow \phi_{s(n)}(s(n)) \text{ опр }  \Leftrightarrow\\
%     &\Leftrightarrow V(n, s(n)) \text{ опр } \Leftrightarrow n \in A
% \end{align*}
% Получили сводимость $ A \leq_m X$.  ............

\section*{Задача №5}
 Пусть у нас есть два коперечеслимых множества перечеслимы,
 которые не пересекаются.
 $A, \; B,\; A \cap B = \varnothing, \; N \setminus A, \; N\setminus B$ -  перечислимые мн-ва
множество С - разрешимо. Запускаем два перечислятора  $\; N \setminus A, \; N\setminus B$.
Мы сможем получить харак. ф-ию множества С. Запускаем два перечислятора  
$i_{N \setminus B} \in \; N \setminus A, \; i_{N \setminus B} \in \;
 N\setminus B,$ если  первым $x = i_{N \setminus B}$ Хар. ф-ия выдает 1, 
 иначе это число выдаст второй перечислятор и хар ф-ия выдаст 0. $\Rightarrow$ C - разрешимо.
 

\section*{Задача №8}
Положим \[V(i, x) = i + x\]
так как $\phi$ - главная $\Rightarrow \exists \text{ тотальная вычислимая ф-ия }s:N \rightarrow N $ такая, что:
\[\phi_{s(i)}(x) = V(i, x)\]
По  теор. Клини о неподвижной точке $\Rightarrow$
\[\exists \; i \; \phi_i(x) = \phi_{s(i)}(x) = V(i,x) = i + x \]
\section*{Задача №7}
Воспользуемся двумя фактами:
\begin{enumerate}
    \item Теорема Поста: 
    Если $A$ и $N \setminus A$ перечеслимы, то $А$ - разрешимо.
    \item Множество значений тотальной ф-ии перечислимое множество.
\end{enumerate}
Пусть 
\begin{equation*}
    A = \{x \;| \;\phi_x(x) определено\}, \; B = N \setminus A  
\end{equation*}
Мы знаем А - неразрешимо и перечеслимо, B неперечесимо, 
иначе у нас будет противоречие с первым пунктом.\\
Допустим, что $A \leq_m B$, тогда:
\[\exists \text{ тотальная вычислимая ф-ия } f: N \rightarrow N, \; x \in A \Leftrightarrow f(x) \in B \]
но в таком случае будет противоречие с пунктом 2. 

\section*{Задача №6}

\begin{align*}
    &A - \text{ перечислимое } \\
    &B = \{i \; | \; \phi_i(2) = \phi_i(3) \text{ и опр}\} \\
    &\omega(n, x) = \begin{cases}
        1 &n \in A\\
        неопр &n \notin A
    \end{cases} 
\end{align*} 
Очевидно, что ф-ия $\omega$ - вычислима, так как А - перечислимое $\Rightarrow$ полуразрешимо 
$\Rightarrow$  существует полухарактеристическая ф-ия $\omega$
а значит   $\exists $ тотальная вычислимая ф-ия  $s: N \rightarrow N$ такая, что:
\[\omega(n,x) = \phi_{s(n)}(x)\] 
для любого x, очев, что при x = 2  и x = 3 и опр, наша ф-ия $\omega = 1$,
мы можем заметить, что $n \in A \Leftrightarrow s(n) = i \in B$ \\
Таким образом мы свели $ A \leq_m B$

\end{document}

