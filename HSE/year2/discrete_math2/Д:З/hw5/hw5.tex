\documentclass[a4paper,12pt]{article} % добавить leqno в [] для нумерации слева

%%% Работа с русским языком
\usepackage{cmap}					% поиск в PDF
\usepackage{mathtext} 				% русские буквы в фомулах
\usepackage[T2A]{fontenc}			% кодировка
\usepackage[utf8]{inputenc}			% кодировка исходного текста
\usepackage[english,russian]{babel}	% локализация и переносы

%%% Дополнительная работа с математикой
\usepackage{amsmath,amsfonts,amssymb,amsthm,mathtools} % AMS
\usepackage{icomma} % "Умная" запятая: $0,2$ --- число, $0, 2$ --- перечисление

%% Номера формул
\mathtoolsset{showonlyrefs=true} % Показывать номера только у тех формул, на которые есть \eqref{} в тексте.

%% Шрифты
\usepackage{euscript}	 % Шрифт Евклид
\usepackage{mathrsfs} % Красивый матшрифт
\usepackage{graphicx}
\usepackage{amsfonts}


%% Перенос знаков в формулах (по Львовскому)
\newcommand*{\hm}[1]{#1\nobreak\discretionary{}
{\hbox{$\mathsurround=0pt #1$}}{}}

%%% Заголовок
\author{Фархат Агаев}
\title{Домашнее задание по дискретной математике №5}
\date{\today}

\begin{document} % конец преамбулы, начало документа

\maketitle
\section*{Задача №1}
Предваренная нормальная форма выглядик так:
\begin{align*} 
    &Q_i \in \{\exists, \; \forall\} \\
    &X_i - \text{ переменные} \\
    &\phi - \text{формула без кванторов} \\
    &Q_1 X_1 \cdots Q_n X_n\phi
\end{align*}
Сколемевская нормальная формула - предваренная нормальная формула 
без квантаров существования.
\begin{align*}
    &\textbf{a)} \exists x \forall y P(x, y) \wedge \forall x \exists y Q(x, y) \\
    &\exists x \forall y P(x, y) \wedge \forall z \exists w Q(z, w) \\
    &\exists x \forall y \forall z \exists w  \, P(x, y) \wedge Q(z, w) - \text{пред. норм. ф.}
\end{align*}
Чтобы избавиться от квантора существования, который стоит до кв. всеобщности,
то достаточно заменить переменную на константу 'c', если стоит после,
то нужен функциональный символ 'f', который принимает переменные, стоящие до и под знаком кв. всеобщности.
\begin{align*}
    \forall y \forall z P(c, y) \wedge Q(z, f(y, z)) -\text{ скол. н. ф.}
\end{align*} 
Повторим процеду0ры для пункта б.
\\
\begin{align*}
    \neg \forall x (\exists y P(x, y) \rightarrow \exists y Q(x, y)) \\
    \exists x (\exists y P(x, y) \rightarrow \exists y Q(x, y)) \\
    \exists x \forall y (P(x, y) \rightarrow \exists y Q(x, y)) \\
    \exists x \forall y \exists z(P(x, y) \rightarrow  Q(x, z)) - \text{ пред. ф.} \\
    \forall y(P(c, y) \rightarrow  Q(c, f(y))) - \text{ скол. н. ф.}
\end{align*}

\section*{Задача №2}
\textbf{a) Условие:}
\begin{enumerate}
    \item $\forall x P(x, f(x))$,
    \item $\forall z Q(x, z)$, 
    \item $\forall x \forall y \forall z (\neg P(x, y) \vee \neg P(y, z) \vee \neg Q(x,z)) $
\end{enumerate}
Для того, чтобы доказать невыполнимость 
набора универсальныъ дизъюнктов, нужно вывести с помощью ИР пустой дизъюнкт.
\\
\textbf{4}. Во втором пункте возьмем $z = f(f(x))$ и получим $ Q(x, f(f(x)))$ \\
\textbf{5}. Третий пункт $y = f(x), z = f(f(x))$ и получим\\
$\neg P(x, f(x)) \vee \neg P(f(x), f(f(x))) \vee \neg Q(x,f(f(x))) $\\
\textbf{6}. Используем ИР для 1 и 4 и получим \\
$\neg P(f(x), f(f(x))) \vee \neg Q(x,f(f(x))$ \\
\textbf{7}. Используем ИР для 6 и 4 и получим \\
$\neg P(f(x), f(f(x))) $ \\
\textbf{8}. Для первого берем  $x = f(x)$ \\
$P(f(x), f(f(x)))$ \\
\textbf{9}. Используем ИР для 7 и 8 и получаем пустой дизъюнкт\\

\textbf{b) Условие:}
\begin{enumerate}
    \item $\forall x \exists y P(x, y)$
    \item $\forall x \forall y \forall z (P(x, y) \wedge Q(y, z) \rightarrow R(x, z))$
    \item $\exists x \forall y \neg R(x, y)$
    \item $\forall \exists y Q(x,y)$ \\
\end{enumerate}

\textbf{Действия:}
\begin{enumerate}
    \item возьмем используя первую формулу (х какая-та константа) $\; P(x, f(x))$
    \item Возьмем используяю вторую формулу  \\
    $\neg P(x, f(x)) \vee \neg Q(f(x), g(f(x))) \vee R(x, g(f(x)))$
    \item $\neg R(x, g(f(x)))$
    \item $ Q(f(x), g(f(x))$ 
    \item Используем ИР для 1 и 2 и получаем \\
    $\neg Q(f(x), g(f(x))) \vee R(x, g(f(x)))$
    \item Используем ИР для 4 и 5 и получаем \\
    $\neg R(x, g(f(x)))$
    \item Используем ИР для 6 и 3 и получаем пустой дизъюнкт\\
\end{enumerate}


\textbf{c) Условие:}
\begin{enumerate}
    \item Доказать общезначимость: \\
    $\forall x \exists y P(x, y) \rightarrow \exists x \exists y P(x, x)$
\end{enumerate}
Чтобы док-ть
общезначимость достаточно док-ть невыполнимость ее отрицания.
\begin{enumerate}
    \item $\forall x \exists y P(x, y)$
    \item $ \forall x \forall y \neg P(x, x) $
\end{enumerate}
\textbf{Действия:}
\begin{enumerate}
    \item $\forall x P(x, f(x)$
    \item пусть $x = f(x)$ \\
    $P(f(x), f(x)$
    \item во второп пункте выше возмем x = f(x) и получим:
    $\neg P(f(x), f(x)$
    \item Очев.
\end{enumerate}
\textbf{d) Условие:} 
Доказать, что из формул:
\begin{enumerate}
    \item $\forall x \exists y(P(x, y) \rightarrow Q(x))$
    \item $\forall x \forall y \forall z (P(x, y) \rightarrow P(x, z))$
    \item $\exists x \exists y P(x, y)$ 
\end{enumerate}
Cемантически следует формулу $\exists x Q(x)$. \\
Для этого нужно доказать невыполнимость данной формулы
\[
    \neg(\forall x \exists y(P(x, y) \rightarrow Q(x)) \wedge
    \forall x \forall y \forall z (P(x, y) \rightarrow P(x, z)) \wedge
    \exists x \exists y P(x, y) \rightarrow \exists x Q(x).)
\]
Как и раньше расскроем отрицание, потом убререм импликацию и выведем пустой дизъюнкт.

\begin{enumerate}
    \item $\forall x \neg P(x, f(x) \vee \forall x Q(x))$
    \item $\forall x \forall y \forall z (\neg P(x, y) \vee P(x, z))$
    \item $P(x, y)$
    \item $\forall x \neg Q(x)$
    \item Используем ИР для 1 и 4: \\
    $\forall x \neg P(x, f(x))$
    \item испоkьзуя 2 пункт получим: \\
    $(\neg P(x, y) \vee P(x, f(x)))$
    \item Используем ИР для 5 и 7 полуаем пустой дизъюнкт.
    $P(x, f(x))$
    
\end{enumerate}

\section*{Задача №5}
\textbf{a)}
($(\mathbb{N}, *, =)$ и $((\mathbb{Z}, *, =)$) изоморфны ли? \\
Очевидно, что нет. так в натуральных числах данное уравнение $x \cdot y \cdot 2 = 2$
Имеет лишь одно ркешение при x = 1, y = 1, в то время как в целых числах два решения при x = 1, y = 1 
и x = -1, y = -1. \\ \\
\textbf{b)}
Да. Предоставлю изоморфизм.
0 переходит в 0, 1 переходит в 3, 2 переходит в 1, 3 переходит в 4, 2 переходит в 2.
То есть проверим напрямую. $x - y = 2.$ \\
$\phi(x) - \phi(y) = 1$
Например $x = 2, y = 0.$
Тогда 
$\phi(x) = 1, \; \phi(y) = 0 $\\ 
1 - 0 = 1.
\section*{Задача №3}
Есть два предиката P, Q;
Они выдают True или False $\Rightarrow$ для любого x, мы можем выдать всего 
4 разных варианта использую два предикита 

[T, T], [F, F], [T, F], [F, T]. 
\\
То есть, если мы разобьем наше n элментное множество на 4 класса,
если размеры классов будут совпадать,
тогда мы сможем
построить изоморфизм переводя из одной модели в другую в нужные классы. То есть чтобы 
модели не были изоморфны нам нужно разбить множество на классы с разным кол-ом элеменитв.
Таким образом мы можем решить задачу с помощью шаров и перегородок поделив на 4 класса (3 перегорадки понадобятся)
\textbf{Ответ:}
\[
  C^n_{n+3}  
\]

\end{document}