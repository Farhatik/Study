\documentclass[a4paper,12pt]{article} % добавить leqno в [] для нумерации слева

%%% Работа с русским языком
\usepackage{cmap}					% поиск в PDF
\usepackage{mathtext} 				% русские буквы в фомулах
\usepackage[T2A]{fontenc}			% кодировка
\usepackage[utf8]{inputenc}			% кодировка исходного текста
\usepackage[english,russian]{babel}	% локализация и переносы

%%% Дополнительная работа с математикой
\usepackage{amsmath,amsfonts,amssymb,amsthm,mathtools} % AMS
\usepackage{icomma} % "Умная" запятая: $0,2$ --- число, $0, 2$ --- перечисление

%% Номера формул
\mathtoolsset{showonlyrefs=true} % Показывать номера только у тех формул, на которые есть \eqref{} в тексте.

%% Шрифты
\usepackage{euscript}	 % Шрифт Евклид
\usepackage{mathrsfs} % Красивый матшрифт
\usepackage{graphicx}
\usepackage{amsfonts}


%% Перенос знаков в формулах (по Львовскому)
\newcommand*{\hm}[1]{#1\nobreak\discretionary{}
{\hbox{$\mathsurround=0pt #1$}}{}}

%%% Заголовок
\author{Фархат Агаев}
\title{Домашнее задание по дискретной математике №6}
\date{\today}

\begin{document} % конец преамбулы, начало документа

\maketitle
\section*{Задача №3}
\textbf{a)}
\[
    \{\forall x P(x, f(x)), \forall x \neg P(x, f(f(x)))\}
\]

Очевидно, модель совместна, пусть носитель $M = \{
    целые \; числа \; без \; нуля\}$, $f(x) = -x $.
    Тогда предикат $P$ - отношение неравно. Очевидно, что верхнее утверждение всегда верно \\ \\
\textbf{b)}
\[\{\forall x P(x, f(x)), \forall x P(g(x), x),
\forall x \forall y \forall z (\neg P(x, z) \vee \neg P(y, z))\}        
\]
С помощью ИР докажем несовместность набора.

\begin{enumerate}
    \item подставляем нужные термы в дизъюнкты и получаем. \\ $P(c, f(c)), P(g(f(c)), f(c)),
    \neg P(g(f(c)), f(c)) \vee \neg P(f(c), c)))$
    \item пользуемся Правилом Резолюций для первого и третьего дизъюнкта из пунтка 1.
    \\ $\neg P(g(f(c)), f(c))$
    \item пользуемся Правилом Резолюций для дизъюнкта из п.2 и второго дизъюнкта из пунтка 1.
    и получаем пустой дизьюнкт 
\end{enumerate}
Cледовательно теория несовместна.


\end{document}