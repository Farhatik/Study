\documentclass[10pt, a4paper]{extarticle}

%% Язык
\usepackage{cmap} % Поиск в PDF
\usepackage{mathtext} % Кириллица в формулах
\usepackage[T2A]{fontenc} % Кодировка
\usepackage[utf8]{inputenc} % Кодировка
\usepackage[english,russian]{babel} % Локализация, переносы


\usepackage{wasysym}


%% Шрифты

% Serif
%\usepackage{euscript} % Шрифт Евклид
%\usepackage{mathrsfs} % Шрифт для математики
\usepackage{libertinus}

% Sans-serif
%\renewcommand{\rmdefault}{cmss}
%\renewcommand{\ttdefault}{cmss}
%\usepackage{sfmath}

% Настройки для xelatex
%\usepackage{polyglossia} % Для выбора языка в xelatex
%\setmainlanguage{russian}
%\setotherlanguages{english}
% Ligatures=TeX is on by default
% https://tex.stackexchange.com/questions/323542/
%\setmainfont[Ligatures=TeX]{Cantarell}
%\newfontfamily{\cyrillicfonttt}{Times New Roman}
%\newfontfamily\cyrillicfont{Cantarell}[Script=Cyrillic]
%\setsansfont[Ligatures=TeX]{Cantarell}
%\newfontfamily\cyrillicfontsf{Cantarell}[Script=Cyrillic]
%\setmonofont{Courier New}
%\newfontfamily\cyrillicfonttt{Courier New}[Script=Cyrillic]

%% Математика
\usepackage{amsmath, amsfonts, amssymb, amsthm, mathtools}
\usepackage{icomma}

% Операторы
\DeclareMathOperator*\plim{plim}
\DeclareMathOperator{\sgn}{sign}
\DeclareMathOperator{\sign}{sign}
\DeclareMathOperator*{\argmin}{arg\,min}
\DeclareMathOperator*{\argmax}{arg\,max}
\DeclareMathOperator*{\amn}{arg\,min}
\DeclareMathOperator*{\amx}{arg\,max}
\DeclareMathOperator{\cov}{Cov}
\DeclareMathOperator{\Var}{Var}
\DeclareMathOperator{\Cov}{Cov}
\DeclareMathOperator{\Corr}{Corr}
\DeclareMathOperator{\pCorr}{pCorr}
\DeclareMathOperator{\E}{\mathbb{E}}
\let\P\relax
\DeclareMathOperator{\P}{\mathbb{P}}
\renewcommand{\le}{\leqslant}
\renewcommand{\ge}{\geqslant}
\renewcommand{\leq}{\leqslant}
\renewcommand{\geq}{\geqslant}

% Распределения
\newcommand{\cN}{\mathcal{N}}
\newcommand{\cU}{\mathcal{U}}
\newcommand{\cBinom}{\mathcal{Binom}}
\newcommand{\cPois}{\mathcal{Pois}}
\newcommand{\cBeta}{\mathcal{Beta}}
\newcommand{\cGamma}{\mathcal{Gamma}}

% Множества
\def \R{\mathbb{R}}
\def \N{\mathbb{N}}
\def \Z{\mathbb{Z}}

% Другое
\newcommand{\dx}[1]{\,\mathrm{d}#1} % Для интеграла: маленький отступ и прямая d
\newcommand{\ind}[1]{\mathbbm{1}_{\{#1\}}} % Индикатор события
\newcommand{\iid}{\mathrel{\stackrel{\rm i.\,i.\,d.}\sim}}
\newcommand{\const}{\mathrm{const}}

%% Изображения
\usepackage{graphicx}
\usepackage{caption}
\usepackage{subcaption}
\usepackage{physics}
\usepackage{wrapfig} % Обтекание рисунков и таблиц текстом
\usepackage{tikz}

%% Таблицы
\usepackage{array, tabularx, tabulary, booktabs}
\usepackage{longtable}  % Длинные таблицы
\usepackage{multirow} % Слияние строк в таблице

%% Cписки
\usepackage{multicol}
\usepackage{enumitem}

%% Гиперссылки
\usepackage{xcolor}
\usepackage{hyperref}
\definecolor{linkcolor}{HTML}{8b00ff}
\hypersetup{colorlinks = true,
			linkcolor = linkcolor,
			urlcolor = linkcolor,
			citecolor = linkcolor}

%% Выравнивание
\setlength{\parskip}{0.5em} % Расстояние между абзацами
\usepackage{geometry} % Поля
\geometry{
	a4paper,
	left=20mm,
	top=20mm,
	right=20mm}
\setlength{\parindent}{0cm} % Отступ (красная строка)
\linespread{1.0} % Интерлиньяж
\usepackage[many]{tcolorbox}  

%% Оформление

\newtcolorbox{rulesbox}[1]{%
	tikznode boxed title,
	enhanced,
	arc=0mm,
	interior style={white},
	attach boxed title to top center= {yshift=-\tcboxedtitleheight/2},
	fonttitle=\bfseries,
	colbacktitle=white,coltitle=black,
	boxed title style={size=normal,colframe=white,boxrule=0pt},
	title={#1}}
	
	

% Красивый серый фон
\usepackage{framed} 
\definecolor{shadecolor}{gray}{0.9}

% Код
\newcommand{\code}[1]{{\tt #1}}

% Колонтитулы
\usepackage{fancyhdr}
\pagestyle{fancy}
\fancyhf{}
\fancyhead[L]{}
\fancyhead[R]{\thepage}

% Разделы и подразделы
\usepackage[sf, sl, outermarks]{titlesec}
\titleformat{\section}{\Large\bfseries\sffamily}{\thesection}{0.5em}{}
\titleformat{\subsection}{\large\sffamily}{\thesubsection}{0.5em}{}

% Содержание
%\usepackage{tocloft}
%\renewcommand{\cftsecfont}{\hspace{4.5em}\normalfont}
%\renewcommand{\cftsubsecfont}{\hspace{5em}\normalfont}
%\renewcommand{\cftsecpagefont}{\normalfont\hfill}
%\renewcommand{\cfttoctitlefont}{\large\normalfont\hfill}
%\renewcommand{\cftaftertoctitle}{\hfill}
%\renewcommand{\cftsecleader}{\cftdotfill{\cftdotsep}}
%\renewcommand{\cftsecafterpnum}{\hspace*{5.5em}\hfill}
%\renewcommand{\cftsubsecafterpnum}{\hspace*{5.5em}\hfill}
%\renewcommand{\cftsecaftersnum}{.}
%\renewcommand{\cftsubsecaftersnum}{.}

%% Комментарии
\usepackage{comment}

%% To-do
\usepackage{todonotes}



%% Литература
\usepackage[backend = biber,
			bibencoding = utf8, 
			sorting = nty, 
			maxcitenames = 4,
			style = numeric-verb]{biblatex}
\addbibresource{lit.bib}
\usepackage{csquotes}


% \usepackage{graphicx}
% \usepackage{color} 
\usepackage[dvipsnames]{xcolor}

\definecolor{mypink1}{rgb}{0.858, 0.188, 0.478}
\definecolor{mypink3}{cmyk}{0, 0.7808, 0.4429, 0.1412}

%% Заголовок
\title{{\normalsize Прикладная статистика в машинном обучении} \\\vspace{0.5em} Лекция №5}
\author{Адыгамов Ильяс 181, Болотин Арсений 182, Агаев Фархат 188}
\date{\rule{15cm}{0.4pt}}


\begin{document}
	
\maketitle
	
\begin{rulesbox}{План лекции}
\begin{enumerate}
	    \item Опишем EM алгоритм на другом языке используя дивергенцию Кульбака-Лейблера
	    \item Изучим Bootstrap: способ построения доверительных интервалов
\end{enumerate}
\end{rulesbox}

\subsection*{\textbf{Обозначения}}
\begin{itemize}
    \item $\theta = (\theta_1, \; \theta_2, \; \dots, \; \theta_n)$ - вектор неизвестных параметров
    \item $x = (x_1,\; x_2,\; \dotsc,\; x_n)$ - наблюдения 
    \item $z = (z_1, \; z_2, \; \dotsc, \; z_n)$ - латентная переменная
\end{itemize}
\subsection*{\textbf{Постановка задачи:}}
Мы хотим максимизировать лографим функции правдоподобия, чтобы найти оценку вектора параметров $\theta$
\[
\max_{\theta} \ln p(x|\theta)
\]
Однако бывает так, что данная функция имеет такой вид, что максимизировать её сложно. 
Поэтому мы заменим данную процедуру на \textbf{EM} алгоритм.
\subsection*{\textbf{EM-алгоритм в общем виде}}
\begin{itemize}
    \item[$\blacksquare$] \textbf{Init}. Задать начальные условя для $\theta_{old} := \theta_{init}$
    \item[$\blacksquare$] \textbf{E-step}. Найти условное распределения латентной переменной $p(z|x, \theta)$
\end{itemize}
\[ Q(\theta, \theta_{old})= \E  [  \ln p (x, z | \theta ) ] \]

\begin{itemize}
    \item[$\blacksquare$] \textbf{M-step}. Максимизируем функцию $Q$ по $\theta$. Получаем $\theta_{new}$. Обновляаем $\theta_{old} := \theta_{new}$
    \item[$\blacksquare$] \textit{Повторяем E-шаг и M-шаг до сходимости}
\end{itemize}



С точки зрения программирования \textbf{M-шаг} довольный легкий, так как есть куча оптимизаторов и максимизировать функцию будет несложно. На \textbf{E-шаге} нужно думать, поэтому заменим на что-то более простое, где нужно будет оптимизировать.


То есть глобльная наша цель поручить нахождение $p(z|x, \theta)$ компьютеру. В этом нам поможет дивергенция Кульбака-Лейблера.
\[
    KL(q||p_{z|x, \theta}) = CE(q || p_{z|x, \theta}) - H(q)
\]
$q$ - распределение кандидат, $\;p_{z|x, \theta}$ - распределение, которое хотим найти.

\subsection*{\textcolor{mypink1}{Напоминание: Дивергенция Кульбака-Лейблера на примере данетки}}
\[
     KL(a || b) = CE(a || b) - H(a)
\]

\begin{itemize}
    \item $a$ - истинное распределение загадывающего данетку
    \item $b$ - распределение, которое предполагает разгадывающий данетку
    \item CE(a||b) - среднее количество вопросов на разгадывание 
    \item H(a)  - среднее количество вопросов в идеальной ситуации (когда разгадывающий знает истинные вероятности, с которыми загадывающий загадывает данетку)
    \item CE(a||b) - H(a)  - лишние вопросы, заданные разгадывающим (в среднем)
\end{itemize}

Также мы знаем, что $KL(a || b) \geq 0$

Первая идея (наивная) давайте попробуем минимизировать дивергенцию Кульбака-Лейблера:
\[
\min_{q}KL(q|| p_{z|x, \theta}) 
\]
Мы знаем, что в таком случае минимум данной функции будет достигаться когда 
\[q^* = p_{z|x, \theta}\]
То есть наивная идея не сработала так как, чтобы выписать $KL$, надо уже знать $p_{z|x, \theta}$. Но, к счастью, сработает лайфхак
\subsection*{\textcolor{blue}{Лайфхак}}
\[
\ln p(x\theta) = KL(q || p_{z|x, \theta}) + LB(q, \theta)
\]
\begin{itemize}
    \item $\ln p(x\theta)$ не зависит от q
    \item $KL(q || p_{z|x), \theta} \geq 0$ 
    \item $LB$ - lower bound
    
\end{itemize}
То есть мы попробуем заменить минимизацию $KL$ на такой 
\textbf{E-шаг:}
\[
LB(q, \theta) \rightarrow \max_{q}, \text{ где ответом будет } q^* = p_{z|x, \theta}
\]
И в этот моменте произойдет чудо: для записи $LB$ не нужно заранее знать $p_{z|x, \theta}$ 
\subsection*{\textcolor{mypink1}{Еще одно маленькое напоминание}}
\[
CE(q||p) = \int q(z) \log_{\frac{1}{2}} p(z)dz = -\int q(z) \log_{2}p(z)dz \text{ (в битах)}
\]
При переводе в наты появится константа
\[
= - \int q(z) \ln p (dz) \cdot C
\]
А теперь распишем $LB$
\[
LB(q, \theta) = \ln p (x|\theta)  - KL(q || p_{z|x, \theta}) = 
\]
\[
= \ln p (x | \theta) - [CE(q||p_{z|x, \theta}) - H(q)] = \ln p(x|\theta) - (-\int q(z) \ln p(z|x, \theta)dz + \int q(z) \ln q(z)dz) = \]
\[
= \int q(z) \ln p(x|\theta)dz + \int q (z) \ln p(z|x, \theta)dz -  \int q(z) \ln q(z)dz
\]
\[
 = \int q(z) \cdot \ln \frac{p(x | \theta) \cdot p(z|x, \theta)}{q(z)}dz  =  \int q(z) \cdot \ln \frac{p(x, z | \theta)}{q(z)}dz
\]
В итоге как мы можем видеть наша функция $LB$ не зависит от $p_{z|x, \theta}$. \smiley

\end{document}